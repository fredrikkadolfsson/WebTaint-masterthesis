\documentclass{../kththesis}
\usepackage{csquotes} % Recommended by biblatex
\usepackage{biblatex}
\usepackage{pgfgantt}

\addbibresource{../references.bib} % The file containing our references, in BibTeX format


\title{Implementing Dynamic Taint Propagation to Enforce Domain Driven Security \\
        \large Specification and Time Schedule}
\author{Fredrik Adolfsson - freado@kth.se}
\email{freado@kth.se}
\supervisor{Musard Balliu}
\examiner{Mads Dam}
\principal{Jonatan Landsberg \& Simon Tardell}
\programme{Master in Computer Science}
\school{School of Computer Science and Communication}
\date{\today}
	

\begin{document}

% Frontmatter includes the titlepage, abstracts and table-of-contents
\frontmatter


\titlepage


\tableofcontents


% Mainmatter is where the actual contents of the thesis goes
\mainmatter



\chapter{Introduction}
1990 was the World Wide Web (Web) founded by Tim Berners-Lee and the creation have cause a huge impact on todays society. \parencite{www} The Web is a good source for information and it connects the world in one unanimous platform. Many bushiness have decided to take advantage of this to gain accessability for their users to the system. But this accessability gain for the wanted user group dose not come without its drawbacks. The accessability is a weakness in the same manner as it is a strength. The Web Application is not only accessible for the wanted user group but for all user groups. Which entails that users who wishes to abuse and/or cause harm to the application have the accessability to do so. 

There are a number of possible attacks that a Web Application is vulnerable to and the attack that might be the most frequently conducted today will probably not be the same as the most performed in the future. The organization Open Web Applications Security Project, mostly known for its shortening OWASP, is a online community which aim to provide knowledge about how to secure Web Applications. \parencite{OpenWebApplicationSecurityProject} OWASP have produced reports about the top 10 security risks with a web application and the latest was published 2017. This report contains information about the ten most common application security risks for that year. Among those security risks is number one Injection Attacks and number seven Cross-Site Scripting. \parencite{OWASP2017, OpenWebApplicationSecurityProject, CrossMichael2007Dgtw}

This thesis will look at the two named security risk and perform evaluations and benchmarks of a possible solution to prevent these kinds of attacks. 



\chapter{Background}
% Description of the area within which the degree project is being carried out (e.g. connection to research/development, state-of-the-art, scientific and/or societal interest).
As said in the previous chapter is Injection Attacks the number one security vulnerability for Web Applications. Injection Attack is a collection name for any attack where the attacker's input changes the intent of the execution. Some possible versions of Injection Attacks is injection of queries that manipulates SQL, NoSQL, OS and LDAP. \parencite{OWASP2017} The most common goals and result of a Injection Attacks are file destruction, lack of accountability, denial of access and data loss. \parencite{Secure_Web} 

Cross-Site Scripting (XSS) in other hand was number seven at OWASP top 10 security vulnerabilities published in 2017. \parencite{OWASP2017} But the origin of the attack goes back to the beginning of the Web and one of the first XSS attacks was conducted just after the release of JavaScript. The attack was conducted trough loading a malicious web application into a frame on the site that the attacker want to gain control of. The attacker could then through malicious scripts from the frame access any content that is visible or typed into the web application. The first prevention for XSS was then introduces trough the standard of Same-Origin Policy. Same-Origin Policy restricts JavaScript to only access content from its own origin. \parencite{FogieSeth2007Xacs, w3csop} 

To prevent these form of security vulnerabilities in web applications have a variety of tools and methodologies been created. One of these is Dynamic Taint Propagation which goal is to prevent possible Injection and Cross-Site Scripting attacks in run time. This is done by marking input variables from sources, which is a marking point where malicious data might enter the system, as tainted through a taint flag attached to the variable. This taint flag follows the variable throughout the application and propagates onto the other variables it comes in contact with. It is possible to detaint (remove the taint flag) a tainted variable but this is only done after the variable have been sanitized trough validation. The taint values are checked in areas called sinks which is a marking for entrypoint to sensitive code such as SQL executions. \parencite{Pan2015, Venkataramani2008} The decision of what to do when a tainted variable tries to pass trough a sink might vary depending on the application. But in general for a Dynamic Taint Propagation tool is the common reaction to stop the execution of the tainted code. But other actions such as logging or raising un alarm is not uncommon. 

But there is not only tools that have been created to help prevent Injection Attacks and XSS. One methodology that have been coined to help secure applications is the programming paradigm Domain Driven Security. Domain Driven Security aim to secure applications by focusing on the core domain models and making certain that validation of the value objects are correct. \parencite{Wilander2009, Johnsson2009}

The question that this thesis will evaluate is if we can combine Dynamic Taint Propagation and Domain Driven Security and develop a tool that enforces the security gains of Domain Driven Security.


\section{Related Work}
\textcite{Stendahl2016} wrote a thesis in 2016 where he evaluated if a Domain Driven Security is able to prevent Injection Attacks and Cross-Site Scripting. He came to the conclusion that there is a security gains towards Injection Attacks and Cross-Site Scripting and the gained security comes from proper validation.

\textcite{Haldar} have written a report about Dynamic Taint Propagation in Java where they try to solves the problem of not properly validating user input. They managed to construct tool that is independent from the Web Applications source code and can see a gain in security. \textcite{Haldar} ran their benchmarks on OWASP's project WebGoat \parencite{webgoat} but acknowledged in their report that benchmarks of real Web Applications need to be tested.

There do exist two Dynamic Taint Propagation tools where Phosphor \parencite{phosphor} is one and Security Taint Propagation \parencite{securityTaint} is another. Both are open source projects and developed for Java applications.


\section{Goal \& Objective}
% The principal's interest. The background of the specific assignment to be conducted. What is the desired outcome (from the principal's side and from the perspective of the degree project)
The goal of this thesis is to implement and benchmark a Dynamic Taint Propagation tool which aims to enforce the security gains of Domain Driven Security. The Dynamic Taint Propagation tools meaning will be to prevent Injection Attacks and Cross-Site Scripting in run time were possible malicious code shall be prevented from executing and logged.

The principal, Omegapoint, is interested in everything that might validate, invalidate, evolve or bring a further value to the programming paradigm Domain Driven Security. The reason for this is because the concept of Domain Driven Security was born and is in development by Omegapoint consultants. Omegapoint also like to se a prototype of a Dynamic Taint Propagation tool which is able to block attacks in run time.



\chapter{Research Question \& Method}
% The QUESTION that will be examined. Formulated as an explicit and evaluable question.
\begin{chapquote}{}
	How can an implementation of a Dynamic Taint Propagation tool enforce the security gains of Domain Driven Security.
\end{chapquote}

\noindent
% (e.g. what does the assignment entail and what are the challenges involved?)
% Preliminary description of, for example, algorithms that will be tested, data that will be used.
The assignment would be to evaluate the implementation of a Dynamic Taint Propagation tool and discuss if it helps to enforce the security gains of Domain Driven Security. The process of this thesis would be to conduct, in order:

\begin{description}  
	\item [Literature Study]
	The literature study is where information relevant to the thesis need to be gathered and presented.

	\item [Tainting \& Detainting]
	This step is the part where tainting and detainting rules are decided. These need to be decided since the next step is the implementation of the Dynamic Taint Propagation tool.

	\item [Implementation]
	The implementation step is where the Dynamic Taint Propagation tool is implemented. Omegapoint have developed a proof of concept product which I will continue my work upon. This tool is developed in and for Java with help of the Javassist \parencite{Javassist} which makes the manipulation of bytecode easier. The proof of concept is developed to check taint on HTTP query strings trough a Spring server.

	\item [Benchmarking]
	This step is where the Dynamic Taint Propagation tool will be benchmarked. The Dynamic Taint Propagation tool should be tested on a larger set of applications to make the result significant. The values that is in focus during the benchmark is the values in the table below. 
	
	\begin{itemize}  
		\item Injection Prevention Rate 
		\item False Positive Rate
		\item Added Time Complexity
	\end{itemize}

	\item [Analysis]
	The analysis step is where the benchmarking results is reflected upon and written into the report.

	\item [Report Writing \& Presentation]
	The last steps is to finalize the report and present the thesis.
\end{description}

\noindent
% How is the work scientifically relevant and what is the hypothesis being tested? How is this hypothesis being tested?
The relevance in the thesis lies in the problem with software security. Since we are going towards an age where digitalization only grows larger is the question about how we can secure our software extremely relevant. The hypothesis is that we can help in the process of enforcing more secure software. But the question is with how much and if there are negative side effects such as to much overhead to the runtime.


\chapter{Evaluation \& News Value}
% How is it determined if the objective of the degree project has been fulfilled and if the question has been adequately answered? Preliminary report on the evaluation method, measures and data.
There should be a discussion and evaluation of the implemented Dynamic Taint Propagation tool. This evaluation should contain well thought comments and observations about the benchmarking result. A comparison/analysis of the possibility for the Dynamic Taint Propagation tool to enforce the security gain of Domain Driven Security shall also be conducted. 

% Why does someone want to read the finished work? And who are these people?
The work should be of interest for anyone wanting to see a gain in security. The core idea is to enforce more secure software through Dynamic Taint Propagation. However, since the relation between Dynamic Taint Propagation and Domain Driven Security will be discussed will the practitioners of Domain Driven Security find it extra interesting.




The benchmark will check the values; injection prevention rate, false positive rate and added time overhead. The tools to use as benchmark is all or some of; OWASP Zed \parencite{zed}, w3af \parencite{w3af} and Loader \parencite{loader}. A discussion whether this tool also helps to enforce the programing paradigm Domain Driven Security is also to be conducted.

\chapter{Pre-study}
% What areas will the literature study focus on?
% How shall the necessary knowledge on background and state-of-the-art be obtained?
% What preliminarily important references have been identified?
% Description of the literature studies. What areas will the literature study focus on? How shall the necessary knowledge on background and state-of-the-art be obtained? What preliminarily important references have been identified?
The literature study will focus on gathering the relevant information needed for the report. These areas are listed in the table below:
	
	\begin{itemize}  
		\item Web Applications
		\item Dynamic Taint Propagation
		\item Domain Driven Security
		\item Injection Attacks
		\item XSS
		\item Javassist
	\end{itemize}

\noindent
Research into JVM modifications must also be included since it is needed for the implementation of the Dynamic Taint Propagation tool. The information will be obtained by researching for relevant books, reports and other possible material. Two of the founders of the concept of Domain Driven Security work at Omegapoint and are accessible for questions. Conduction interviews with the founders might be of interest.



\chapter{Conditions \& Schedule}
\section{Resources}
% List of the resources expected to be needed to solve the problem (unless the degree project involves investigating what equipment should be used). This can be technical equipment, but also experiment and interview subjects.
To save some time will the development of the Dynamic Taint Propagation tool continue on the work that Simon Tardell have started. Which is a tool developed in and for Java with help of the Java library Javassist \parencite{Javassist}. Applications to evaluate the implementation is also of need. The thesis is at the moment aimed towards web applications which means that a number, 10 should be sufficient, of web applications need to be gathered. Omegapoint have some internal systems which could be used. Other usable web applications can be found on open source platforms.


\section{Limitations}
% Defined limitations on what is to be done (so that it is clear what is not included in the degree project)
\begin{itemize}  
	\item The Dynamic Taint Propagation tool dose not have to be a production ready. The goal is to develop a prototype.
	\item Web Applications is the targeted applications.
	\item The scope of the thesis will not contain Static Taint Propagation.
	\item The tool is developed in Java with Javassist.
\end{itemize}


\section{Company Supervisor}
% Describe the way in which the principal will be involved in the project and what the external supervisor has undertaken to do (e.g. in terms oef discussion, implementation, report reading).
\begin{itemize}
	\item \textbf{Jonatan Landsberg:} Will assist with supervision on the academic part if the thesis.
	\item \textbf{Simon Tardell:} Supervisor in the technical parts of the thesis. He is also the author of the first draft of the Dynamic Taint Propagation tool which this thesis will continue its work upon. 
\end{itemize}


\section{Time Plan}
% Show not only the order in which elements will be performed, but also the scope of elements and when they will be performed (calendar week number or date). The schedule should contain clear intermediate goals.
Below is my time plan for the Masters Thesis. The goal is to continuously, throughout all phases, add to the report. But I have also reserved a couple of weeks in the end for writing the report. I believe that this time can be used to add to or rewrite sections if needed. \\ \\

% today=5
\begin{ganttchart}[vgrid, hgrid]{1}{20}
	\gantttitle{Maste Thesis Time Plan (Weeks)}{20} \\
	\gantttitlelist{1,...,20}{1}\\
	\ganttbar{Start-Up}{1}{3} \\
	\ganttbar{Literature Study}{4}{6} \\
	\ganttbar{Tainting \& Detainting}{7}{7} \\
	\ganttbar{Implementation}{8}{13} \\
	\ganttbar{Benchmarking \& Analysis}{14}{16} \\
	\ganttbar{Report Writing}{17}{19} \\
	\ganttbar{Presentation}{20}{20}
\end{ganttchart}

\printbibliography[heading=bibintoc] % Print the bibliography (and make it appear in the table of contents)

\end{document}
