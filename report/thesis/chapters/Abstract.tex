\begin{abstract}
	The internet is a source of information and it connects the world through a single platform. Many businesses have taken advantage of this to share information, communicate with customers, and create new business opportunities. However, this does not come without drawbacks. There are several potential attacks causing harm to web applications.

	The thesis implemented a dynamic taint tracker, named WebTaint, to detect and prevent confidentiality and integrity vulnerabilities in Java-based web applications. We evaluated to what extent WebTaint can combat integrity vulnerabilities. The questions of what advantages and disadvantages are introduced by using the application and if the application were capable of being integrated into production services were answered.

	The results show that WebTaint helps to combat SQL Injection and Cross-Site Scripting attacks. However, there are drawbacks in the form of additional time and memory overhead. The implemented solution is therefore not suitable for time or memory sensitive domains. A use case for WebTaint is in test environments where security experts utilize the taint tracker to find TaintExceptions through manual or automatic attacks.
\end{abstract}



\begin{otherlanguage}{swedish}
	\begin{abstract}
		Internet är en informationskälla och förbinder världen genom en enda plattform. Många företag har utnyttjat detta för att dela information, kommunicera med kunder och skapa nya affärsmöjligheter. Detta kommer dock inte utan nackdelar. Det finns flera potentiella attacker som kan skada webapplikationer. 

		I avhandlingen implementerades en dynamic taint tracker, namngett WebTaint, för att förhindra sekretess och integritetsproblem i Java-baserade webbapplikationer. Vi utvärderade i vilken utsträckning WebTaint kan bekämpa integritets sårbarheter. Frågorna om vilka fördelar och nackdelar som infördes genom att använda applikationen samt om applikationen kunde integreras i produktionstjänster besvarades. 
		
		Resultaten visar att WebTaint hjälper till att bekämpa SQL Injection och Cross-Site Scripting-attacker. Det finns dock nackdelar i form av extra åtgång av tid och minne. Den implementerade lösningen är därför inte lämplig för tids- eller minneskänsliga domäner. Ett användningsfall för WebTaint är i testmiljöer där säkerhetsexperter använder taint trackern för att hitta TaintExceptions genom manuella eller automatiska attacker.
	\end{abstract}
\end{otherlanguage}