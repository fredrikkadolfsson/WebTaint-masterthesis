\begin{abstract}
	The internet is a source of information, and it connects the world through a single platform. Many businesses have taken advantage of this to share information, communicate with customers, and create new business opportunities. However, this does not come without drawbacks. There are several potential attacks causing harm to web applications. The attack most frequently conducted today will probably not be the same as the most performed in the future.

	The thesis implements and evaluates a dynamic taint tracker, called WebTaint, to prevent confidentiality and integrity vulnerabilities in Java-based web applications. The research questions are to what extent can dynamic taint tracking combat confidentiality and integrity vulnerabilities? What drawbacks are there or is it capable enough of bundling into production services? 

	The results show that WebTaint helps to combat SQL Injection and Cross-Site Scripting attacks. However, there are drawbacks in the form of additional time and memory overhead. Which in this case is quite significant. The implemented solution is therefore not suitable for time och memory sensitive domains.
\end{abstract}



\begin{otherlanguage}{swedish}
	\begin{abstract}
		Internet är en informations källa, och förbinder världen genom en enad plattform. Många företag har utnyttjat detta för att dela med sig av information, kommunicera med kunder och skapa nya affärsmöjligheter. Detta kommer dock inte utan nackdelar. Det finns flera potentiella attacker som kan påverka webapplikationer.  

		Avhandlingen implementerar och utvärderar en dynamisk taint tracker, kallad WebTaint, för att förhindra sekretess och integritets sårbarheter i Java-baserade webbapplikationer. Forskningsfrågorna är i vilken utsträckning Dynamic Taint Tracking kan bekämpa sekretess och integritets sårbarheter? Vilka nackdelar finns det eller är implementationen tillräckligt för att levereras tillsammans med produktionstjänster? 
		
		Resultaten visar att WebTaint hjälper till att bekämpa SQL Injektions- och Cross-Site Scripting-attacker. Det finns dock nackdelar i form av extra tid och minnes åtgång. Vilken i detta fall är ganska signifikant. Den implementerade lösningen är därför inte lämplig för tid och minnekänsliga domäner.      
	\end{abstract}
\end{otherlanguage}