\begin{abstract}
	The internet is a source of information and it connects the world through a single platform. Many businesses have taken advantage of this to share information, to communicate with customers, and to create new business opportunities. However, this does not come without drawbacks as there exists an elevated risk to become targeted in attacks.

	The thesis implemented a dynamic taint tracker, named WebTaint, to detect and prevent confidentiality and integrity vulnerabilities in Java-based web applications. We evaluated to what extent WebTaint can combat integrity vulnerabilities. The possible advantages and disadvantages by using the application is introduced as well as an explication whether the application was capable of being integrated into production services.

	The results show that WebTaint helps to combat SQL Injection and Cross-Site Scripting attacks. However, there are drawbacks in the form of additional time and memory overhead. The implemented solution is therefore not suitable for time or memory sensitive domains. WebTaint could be recommended for use in test environments where security experts utilize the taint tracker to find TaintExceptions through manual and automatic attacks.
\end{abstract}



\begin{otherlanguage}{swedish}
	\begin{abstract}
			Internet är en informationskälla och förbinder världen genom en enda plattform. Många företag har utnyttjat detta för att dela information, kommunicera med kunder och skapa nya affärsmöjligheter. Detta kommer emellertid inte utan nackdelar, eftersom det finns en förhöjd risk att bli måltavlor i attacker.
			
	I avhandlingen implementerades en dynamic taint tracker, namngett WebTaint, med uppgift att förhindra sekretess och integritetsproblem i Java-baserade webbapplikationer. Vi utvärderade i vilken utsträckning WebTaint kan bekämpa integritets sårbarheter. De möjliga fördelarna och nackdelarna med användning av applikationen introduceras såväl som en förklaring om applikationen kunde integreras i produktionstjänster.
			
			Resultaten visar att WebTaint hjälper till att bekämpa SQL Injection och Cross-Site Scripting-attacker. Det finns dock nackdelar i form av extra åtgång av tid och minne. Den implementerade lösningen är därför inte lämplig för tids- eller minneskänsliga domäner. Ett användningsfall för WebTaint är i testmiljöer där säkerhetsexperter använder taint trackern för att hitta TaintExceptions genom manuella och automatiska attacker.
	\end{abstract}
\end{otherlanguage}