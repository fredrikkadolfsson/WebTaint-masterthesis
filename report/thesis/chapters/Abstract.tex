\todo[inline]{Todo: Uppdatera till senaste (SISTA SOM GÖRS)}
\begin{abstract}
	The internet is a source of information, and it connects the world through a single platform. Many businesses have decided to take advantage of the web platform to share information and communicate with customers. However, this does not come without drawbacks. There are several potential attacks can cause harm to a web application. The attack most frequently conducted today will probably not be the same as the most performed in the future. 2017 was Injection Attacks and Cross-Site Scripting among the ten most frequently conducted attacks.

	This thesis implements and evaluates a dynamic taint tracker to prevent confidentiality and integrity vulnerabilities in Java-based web applications. Does dynamic taint tracking enforce the same security gains as Domain-Driven Security?

	The results show that dynamic taint tracking helps to combat Injection and Cross-Site Scripting attacks just as Domain-Driven Security. However, there are drawbacks in the form of additional time and memory overhead. Which in this case is quite significant.
\end{abstract}



\begin{otherlanguage}{swedish}
	\begin{abstract}
		\todo[inline]{Todo: Uppdatera till senaste (SISTA SOM GÖRS)}
	\end{abstract}
\end{otherlanguage}