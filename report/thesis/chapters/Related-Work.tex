\chapter{Related Work}
\label{RW}
\textit{This chapter presents the related work in the field. It addresses the areas of dynamic and static taint tracking for Java as well as other related platforms.}

\textcite{Haldar} has written a report about dynamic taint tracking for Java applications where they tried to solve the problem of correct user input validation. They built a tool that is independent of the web applications source code and the results from using the tool were proven to prevent code injection. \textcite{Haldar} evaluated the taint tracker on OWASP WebGoat \parencite{webgoat} but acknowledged that benchmarks of real-world web applications are needed. Their application implemented taint tracking for Strings by adding a taint flag and altered the methods to propagate the taint into the String class file. \textcite{Haldar} implementation cant be found.

Another implementation of a dynamic taint tracker for Java applications is Phosphor \parencite{phosphor} created by \textcite{BellJ.2014PIdd}. Phosphor was developed with the help of the Java bytecode manipulation library ASM \parencite{asm}. The application addresses taint tracking for primitives and arrays by introducing shadow variables. A shadow variable is a variable holding the taint flag for an un-instrumentable object. The shadow variable is injected into the application and placed next to each primitive and array. Each method in the application is also instrumented to pass shadow variables together with the un-instrumented object. Phosphor does however not support detainting of variables \parencite{BellJ.2014PIdd}. 

A third implementation of a dynamic taint tracker for Java applications is Dynamic Security Taint Propagation \parencite{securityTaint} constructed with the help of the Java library AspectJ which enables aspect-oriented programming in Java \parencite{aspectj}. Dynamic Security Taint Propagation only propagates taint for the String, StringBuffer, and StringBuilder classes. The tool relies on aspect-oriented events that trigger the taint propagation, tainting data from sources, and assertions that ensure that tainted values do not pass through sinks.

In addition to dynamic taint trackers for Java applications, there exist dynamic taint trackers for other platforms. One of these is TaintDroid which is constructed for Android smartphones and aims to prevent privacy violations \parencite{EnckWilliam2014Taif}. TaintDroid uses a shadow memory which reduces the memory overhead of the application. \textcite{HsiaoS.W.2014PRse} has as well conducted further work on top of TaintDroid where they created a security scheme called PasDroid. PasDroid enables users to gain full control over the management of possible information leaks.

There exist several static taint trackers, e.g., FlowDroid by \textcite{ArztS.2014FPcf}. FlowDroid computes data flows for Java and Android applications. The static tracker is built with the help of the Soot framework \parencite{soot} providing analysis tools used to find possible vulnerabilities. FlowDroid is evaluated in comparison to AppScan and Fortify. The results show that FlowDroid has a higher true positive and lower false positive rate.