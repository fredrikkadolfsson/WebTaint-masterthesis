\chapter{Related Work}
\textcite{Stendahl2016} and \textcite{Arnor2016} have both, in their thesis's, concluded that Domain-Driven Security help to prevent security vulnerabilities into an application. \textcite{Stendahl2016} thesis, written in 2016, evaluated if Domain-Driven Security can prevent Injection Attacks and Cross-Site Scripting. He reasoned that there is a security gain towards Injection Attacks and Cross-Site Scripting by following the Domain-Driven Security methodology. The gained security comes from proper validation of variables before propagating the data into the value primitives. \textcite{Arnor2016} followed similar reasoning where he discussed the mitigation of DDoS attacks by using Domain-Driven Security.

\textcite{Haldar} has written a report about Dynamic Taint Tracking in Java where they try to solve the problem of not correctly validating user input. They managed to construct a tool that is independent of the web applications source code and the results from using the tool is gain in security. \textcite{Haldar} ran their benchmarks on OWASP’s project WebGoat \parencite{webgoat} but acknowledged in their report that benchmarks of real-world web applications need conducting.

It exists two Dynamic Taint Tracking tools, Phosphor \parencite{phosphor} and Dynamic Security Taint Propagation \parencite{securityTaint}. Both are open source projects and developed for Java applications. Phosphor does however not support sanitizers, and Dynamic Security Taint Propagation cant build from its source code.

The construction of Phosphor \parencite{phosphor} is done in the Java bytecode manipulation library ASM \parencite{asm}. Phosphor, based on the research conducted in the thesis, is the current state of art application in Dynamic Taint Tracking. The application solved tracking of primitive and array taint by introducing shadow variables. A shadow variable is a variable holding the taint value for a un-instrumentable object. The shadow variable is instrumented into the application and placed after each primitive and array. Each method in the application is also instrumented to pass shadow variables together with the un-instrumentable object. This solution makes Phosphor capable of tracking taint for all Java data types \parencite{BellJ.2014PIdd}.

Dynamic Security Taint Propagation \parencite{securityTaint} is constructed with the help of the Java library AspectJ which enables aspect-oriented programming in Java \parencite{aspectj}. Dynamic Security Taint Propagation only propagates taint for the String, StringBuffer, and StringBuilder classes. The tracking works by creating aspect-oriented events that trigger the tracking of taint, tainting sources and assertions that tainted values do not pass through sinks.