\chapter{Discussion}
\label{Discussion}
\textit{This chapter contains discussions regarding the implemented dynamic taint tracker, named WebTaint, and how well it performs. The chapter starts with a general discussion. This is then followed by \textit{\nameref{propagation}} discussion. The last two sections are discussions about \textit{\nameref{sss}} and the \textit{\nameref{methev}}.}

By looking at the results in the previous chapter we see a clear indication of that WebTaint is capable of detecting security vulnerabilities. The Stanford SecuriBench Micro has a 100\% prevention rate by using WebTaint, and the other three applications have 75.5\%, 75\%, and 73.3\%. Making the average across the four to be 81\%. This indicates a significant impact in combating integrity vulnerabilities. The prevention rate could also be increased by the further development of the application where taint tracking support is enabled for charArrays and byteArrays, which however were not implemented due to time limitations during this thesis.

Despite, the increase in security might not in the end be worth it if significant drawbacks are implicated. From the overhead results can we see the use of WebTaint introduces overhead. This overhead comes from the Java Agent instrumenting the classes and the added operations to propagate taint. The application domains where time and memory usage is not a problem would therefore not suffer from the introduced overhead. However, web applications need fast response times to provide a good user experience. This causes them to be time sensitive and that is a reason why WebTaint is not suitable to be included in production systems.

The most significant impact on time overhead comes from the startup phase of the application where the Java Agent instruments classes files. Instrumentation of a class file happens only once, and it is done the first time the Class Loader loads the file. This means that applications executed for an extended period and reuses a smaller set of class files are less affected by the time overhead. It is shown in Figure \ref{fig:Time} where Avrora's and Batika's time overhead are 137.2\% respectively 432.2\% compared to Tradebeans and Tradesoaps 26.3\% respectively 14.7\%.

The memory overhead tells a different story. The two most extended executions, together with almost every execution have the same memory overhead as the average which is 142.7\%. It is only the Eclipse, the H2 and the Jython DaCapo tests that have significantly fewer numbers. It is hard to interpret why these are significantly less. One guess would be that they significantly use a smaller amount of strings and therefore are not affected by the added taint flag and other help functions instrumented into the String, StringBuilder, and StringBuffer classes.



\section{Taint Propagation}
\label{propagation}
Due to time issues, only the classes String, StringBuilder, and StringBuffer were implemented to support taint propagation. The limitation is justifiable as these are the most important classes for taint tracking when securing web applications. However, there is important to take into account the risk of losing the tracking of taint since some libraries use charArrays and byteArrays for String operations. The results prove that the implemented classes have a significant influence on the outcome. Nevertheless, the optimal solution would be with complete integration for all data types in Java. Just like the Phosphor, but with the ability to sanitize variables.



\section{Sources, Sinks \& Sanitizers}
\label{sss}
During the planning phase of the thesis was the task of defining sources, sinks and sanitizers believed to be a minor task. Taint trackers depend on these and it is essential to dedicate work to define these correctly. However, this was a time-consuming task that was out of the scope of the thesis. The solution was to compile definitions of sources, sinks, and sanitizers from sources found online. These consisted of bloggers putting together what they believed some of the sources, sinks, and sanitizers should be.

The optimal solution to define sources, sinks and sanitizers would be extensive research where lists for each library, framework, and deployment utility used by Java-based web applications were compiled. These lists could then be used depending on what functionality the implemented web application is using. The best situation would be that every developer of Java libraries, frameworks, and deployment utilities compiled lists for their implementations.

Another thing of interest would be to introduce multiple taint types. Multiple taint types would lead to a more advanced taint tracking where data from a specific type of sources are sanitized with the correct type of sanitizer. This would reduce the risk of mistakenly using incorrect sanitizers and also ensure better protection of WebTaint.



\section{Methodology of Evaluation}
\label{methev}
The objective of the thesis was from the beginning to implement a dynamic taint tracker and benchmark it in comparison to the Dynamic Security Taint Propagation and Phosphor. This was however not possible since the prior was not able to build from the source code and the other problem was that the Phosphor does not support sanitation of variables. Making the use case for Phosphor not applicable in comparison to WebTaint. It is hard to estimate how well the implemented tool performed when a comparison was not possible to be conducted. However, the results prove that the implementation is of use.