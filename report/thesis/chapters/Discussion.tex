\chapter{Discussion}
\todo[inline]{Todo: Results show that memory overhead is large in some areas. this could be optimized and probably lowered. But this is not a good solution for memory sensitive domains.

The largest added overhead commes from the instrummentation of application during rutime. Meaning that first time a class is imported will be slower but the second will be almost as fast as None. This is proben by treadbeans and tradesoap. This could be solved by instrummenting the application beforehand.
}

\section{Domain-Driven Security}
\todo[inline]{Todo: DDS is proven in earlyer report to combat Injection and XSS. Results prove that number of attacks are lowered by DTT. If only validating in domain primitives in DDS will missing using dp be catastrofal. DTT will still notify missing validation. 

To enforce usage of DDS could sanitation of datatypes only be conducted in constructors. This would force the user to only use domain primitives. 
}

\section{Sources, Sinks and Sanitizers}
\todo[inline]{Todo: Discuss the complexity with declaring SSS. How Should this be solved? Subscribe to list depending on used libraries. Define preset values that is used instead of taint. 
}


\section{Methodology of Evaluation}
\todo[inline]{Todo: Compare to similar applications. Phosphor dose not support detainting. Meaning not applicable as Dynamic Taint Tracker for applications in production where the goal is to halt the execution of taint exceptions.
}