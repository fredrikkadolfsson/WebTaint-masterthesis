\chapter{Future Work}
\label{FutureWork}
There is some work needed to be done before WebTaint can take place as an adequate solution to secure web applications from security vulnerabilities. One thing necessary to do is to finalize the comprehensive work regarding sources, sinks, and sanitizers to ensure correct usage. It would also be of interest to implement the use of different types of sources, sinks, and sanitizers. Implementing different types would allow for an advanced taint tracking where sanitizers only capable of sanitizing one specific type of data and do not mark other types as safe. 

Optimizations of WebTaint's execution time and memory usage is also in need since it was not prioritized during this thesis. Improvements minimizing the introduced overhead would make WebTaint useful in time- or memory sensitive domains. WebTaint would also benefit by enhancing the coverage of data types supporting taint tracking. The two most important data types, not currently supported by WebTaint, are charArrays and byteArrays. 

Another possible WebTaint enhancement is to implement support of implicit taint flows. WebTaint does at the moment only support explicit flows where taints propagate if the calculated variables are directly dependent on a tainted variable. For example would $ x $ in $ x = y + 1 $ become tainted when $ y $ is tainted. The implicit flow would enable taint to propagate implicitly. A example of this is that $ x $ would be tainted in $ if (y) x = 1 $ when $ y $ is tainted.