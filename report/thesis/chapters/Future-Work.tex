\chapter{Future Work}
\label{FutureWork}
There is work needed to be done before WebTaint can take its place as a solution to secure web applications. One of these is to conduct comprehensive work about sources, sinks, and sanitizers to ensure correct usage. It would also be of interest to implement the use of different types of sources, sinks, and sanitizers. Allowing for a more detailed taint processing where sanitizers only capable of sanitizing a specific type of data do not mark other types of data as detainted. 

Work towards optimization of the applications is also in need. Bot towards minimizing the added time overhead as well as the added memory overhead. To enhance the coverage of the tool should expansions of data types supporting tracking of taint carry out as well. The two most important data types are char and byte arrays. 

Another way to enhance the coverage of WebTaint is to develop the application further to support implicit taint flow. WebTaint does at the moment only support explicit taint flows where taint propagates if the resulting variables are directly dependent on the tainted variables. For example would $ x $ in $ x = y + 1 $ become tainted if y is tainted. Implicit taint flow would enable taint to propagate implicitly where an example is that $ x $ would be tainted if $ if (y) x = 1 $ and $ y $ is tainted.

There is also the need for further research and evaluation for the field and the implemented taint tracker. Trial runs where WebTaint runs for a more extended period would be of interest. The interest lies in getting an insight on how the percentage time and memory overhead changes compared to the results of this report.