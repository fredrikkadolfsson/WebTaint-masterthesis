\chapter{Result}
This chapter presents the results from the conducted evaluation. Appendix \ref{appendix:raw} contains raw data and metrics over data that may not been presented in this chapter.

\todo[inline]{Todo: Write suppary of what to come in chapter. }

\section{Performance Overhead}
The results from benchmarking the application on DaCapo Benchmark Suit \parencite{dacapo} is seen in Figure \ref{fig:Time} and \ref{fig:Memory}. Both graphs are constructed to show the added overhead of running the applications with Dynamic Taint Tracking activated. The graphs are conducted based on the data in Table \ref{TimeTable} and \ref{MemoryTable}.


\subsection{Time}
Figure \ref{fig:Time} displays the results of the average time overhead per application. The results show that the application with the least average time overhead was Tradesoap where 14.7\% was added. The largest application however, was Fop with an additional of 426.2\%. The average overall overhead is 142.1\%.

From the Tables \ref{TimeTable} can we as well see that the variation between the minimum and maximum time will for all applications be 437.8\%.

\begin{figure}[H]
	\centering
	\includegraphics[width=\textwidth]{images/Time.png}
	\caption{Average Added Time in Microseconds}
	\label{fig:Time}
\end{figure}


\subsection{Memory}
Figure \ref{fig:Memory} displays the results of the average memory overhead per application. The results show that the application with the least average memory overhead was Eclipse where 5.5\% was added. The largest application however, was Fop with an additional of 277.8\%. The average overall overhead is 127.2\%.

From the Table \ref{MemoryTable} can we as well see that the variation between the minimum and maximum memory will for all 305.3\%.

\begin{figure}[H]
	\centering
	\includegraphics[width=\textwidth]{images/Memory.png}
	\caption{Average Added Memory in Kilobytes}
	\label{fig:Memory}
\end{figure}



\section{Applications}
The results from running the implemented Dynamic Taint Tracker on a set of Java web applications can be seen in Tables \ref{table:MicroTable}, \ref{table:InsecureTable} and \ref{table:SnipSnapTable}. 

In Table \ref{table:MicroTable} and \ref{table:InsecureTable} where Stanford SecuriBench Micro \parencite{securiBenchMicro} respectively Insecure \parencite{insecure} have been benchmarked can we see that by using the implemented Dynamic Taint Tracker have the applications security vulnerabilities been prevented with 100\%. 


\begin{table}[H]
  \centering
  \caption{Security Vulnerabilities Found in Stanford SecuriBench Micro}
  \label{table:MicroTable}
  \resizebox{\columnwidth}{!}{
    \begin{tabular}{rcc}
                                                                & \textbf{None} & \textbf{Dynamic Taint Tracking} \\
      \textbf{SQL Injection (High (Medium))}                    & 20             & 0                               \\
      \textbf{Cross Site Scripting (Reflected) (High (Medium))} & 2              & 0                               \\
      \textbf{Cross Site Scripting (Reflected) (High (Low))}    & 69             & 0                               \\
      \textbf{Web Browser XSS Protection Not Enabled}           & 1              & 0                              
    \end{tabular}
  }
\end{table}


\begin{table}[H]
  \centering
  \caption{Security Vulnerabilities Found in Insecure}
  \label{table:InsecureTable}
  \resizebox{\columnwidth}{!}{
    \begin{tabular}{rcc}
                                                                & \textbf{None} & \textbf{Dynamic Taint Tracking} \\
      \textbf{SQL Injection (High (Medium))}                    & 4              & 0                               \\
      \textbf{Cross Site Scripting (Reflected) (High (Medium))} & 2              & 0                              
    \end{tabular}
  }  
\end{table}


The SnipSnap \parencite{snipsnap} application have seen improvements of vulnerabilities where only 10.6\% of the security vulnerabilities are left after activating the Dynamic Taint Tracker.


\begin{table}[H]
  \centering
  \caption{Security Vulnerabilities Found in SnipSnap}
  \label{table:SnipSnapTable}
  \resizebox{\columnwidth}{!}{
    \begin{tabular}{rcc}
                                                                & \textbf{None} & \textbf{Dynamic Taint Tracking} \\
      \textbf{SQL Injection (High (Medium))}                    & 10             & 1                               \\
      \textbf{Cross Site Scripting (Reflected) (High (Medium))} & 149            & 16                              \\
      \textbf{CRLF Injection Medium (Medium)}                   & 2              & 0                              
    \end{tabular}
  }  
\end{table}
