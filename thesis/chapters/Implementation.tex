\chapter{Implementation}


\section{Policies}


\subsection{Confidentiality}


\subsection{Integrity}


\section{Notable Problems}
One of the first problems that was introduced during the development phase of the application is that some classes can't be instrumented during runtime. More precisely, the classes that the JVM relies on can't be instrumented in realtime. But these is a solution to this. The solution is to create a JAR file with statically modified versions of the classes. In this case is the String class one of these. This JAR file can then be loaded through the option Xbootclasspath/p that appends the JAR file to the front of the bootstrap path. Forcing the JVM to use our modified versions of classes \parencite{xboot}. Because of this limitation were the decision of instrumenting as many classes as possible statically. This is to keep the code consistent.

Another problem is that primitives can't be instrumented. This causes a problem since it opens the ability to miss propagation of tainted data if they ever pass through a byte- or char-array. The solution that can solve this is to create shadow variables that lie in the closest class or objective to the byte- or char-array. This shadow variable will contain the taint.


\section{Software Architecture}
