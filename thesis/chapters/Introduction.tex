\chapter{Introduction}
% Frågeställningen är lätt att identifiera, projektets syfte och mål är tydliga.
% Problemet (och studentens bidrag) är tydligt avgränsat, dess relevans är motiverad och satt i ett sammanhang.
One of the greatest strengths with deploying applications on the World Wide Web is that they are accessible from everywhere where there exists a internet access. This is sadly one of its greatest weaknesses as well. The applications are easily accessible for people who whishes to abuse or cause them harm. Among the number of security risks that a web application is vulnerable to is two of the more common Injection Attack and Cross-Site Scripting. \parencite{OpenWebApplicationSecurityProject, CrossMichael2007Dgtw}

To minimize the risk of accidentally introducing security flaws in to the application have a variety of tools and methodologies been created. One of these is Dynamic Taint Propagation which marks input from the user as tainted through a taint variable attached to the input. This taint variable follows the input throughout the application and propagates onto the other variables it comes in contact with. It is possible to detaint the input but this is only done after the input have been validated. The taint value is later checked in sensitive areas through something called sinks. Execution is halted if a tainted variable is detected trying to enter the sensitive area through the sink. \parencite{Pan2015, Venkataramani2008} One of the methodologies that have been coined is the programming paradigm Domain Driven Security. Domain Driven Security aim to secure applications by focusing on the core domain models and making certain that validation of the value objects are correct. \parencite{Wilander2009, Johnsson2009}

The following sections of the chapter will aim to specify the why and how behind the conduction of the thesis. It starts with a section of \textit{Definitions} followed by \textit{Problem} description and explanation of the thesis \textit{Aim}. These sections is then followed by \textit{Related Work} in the field and a \textit{Delimitations} section. Lastly, is there a section about the \textit{Methodology} behind the thesis.


\section{Definitions}
Throughout this thesis will a consistent terminology be used. Some terms might be new or might just be usable in different ways. To remove confusion will a list of definitions be defined in this section.

\begin{definition}{Application} 
	is a computer process which is constructed to solve one or more tasks for the user.
\end{definition}

\begin{definition}{Web Application} 
	is a application deployed with accessability from the web.
\end{definition}

\begin{definition}{Taint}
	is denoting a flag that is attached to values indicating that the value might be of possible harm to the application.
\end{definition}

\begin{definition}{Detaint}
	denotes the process of removing the taint flag from a value and therefore marking the value as safe to the application.
\end{definition}

\begin{definition}{Domain}
	% Information sida 67 Secure by Design	
	is explained in Secure by Design \parencite{sbd2018} as part of the real world where something happens.
\end{definition}

\begin{definition}{Domain Model}
	% Information sida 67 Secure by Design	
	is a fraction of the domain where each model have a specific meaning.
\end{definition}


\section{Problem}
\hfill \\
\begin{chapquote}{}
	How can an implementation of a Dynamic Taint Propagation tool enforce the security gains of Domain Driven Security.
\end{chapquote}

\noindent
Is it possible to achieve the security gains of Domain Driven Security trough applying Dynamic Taint Propagation to web applications. What would the possible drawbacks and advantages be.


\section{Aim}
The aim of this thesis is to implementation and evaluate a Dynamic Taint Propagation tool and discuss if it helps to enforce the security gains of Domain Driven Security. The benchmark will check the values; injection prevention rate, false positive rate and added time overhead. 


\section{Related Work}
Previous work that have been conducted to Dynamic Taint Propagation is for example the work of \textcite{Haldar} and \textcite{Zhao2016}. Both reports have implemented a Dynamic Taint Propagation tool for web deployed Java applications where the manipulation of the Java bytecode is done trough Javassist.

Domain Driven Security is somewhat more unknown, compared to Dynamic Taint Propagation, and there are not many reports that discusses and analyses the domain. But one thesis that is noticeable, is the thesis of \textcite{Stendahl2016}.


\section{Delimitations}
The thesis will focus on security risks of web applications. Even though other application areas might be vulnerable to the same kind of vulnerabilities have this delimitation been drawn to keep the scope small and precise. The thesis dose nether create nor benchmark any form of Static Taint Analyser. 

The Dynamic Taint Propagation tool is created in and for Java applications with the help of the Java library Javassist. No comparison of different bytecode manipulators will be conducted.

\section{Methodology}
The methodology of this thesis is a combination of quantitative and qualitative research. The first part of the thesis, which consists of the literature study followed by reasoning about tainting and detainting rules, is based on quantitative research. The Second part of the thesis is to evaluate the implemented Dynamic Taint Propagation tool. This is done in a quantitative manner where benchmarks evaluating performance and security gain functionality will be conducted.
