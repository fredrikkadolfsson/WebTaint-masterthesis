\chapter{Background}


\section{Web Applications}

\section{Injection}
\subsection{Cross-site Scripting}
\subsection{SQL}

\section{Taint Propagation}
\subsection{Dynamic}
\subsection{Static}

\section{Domain Driven Design}
There exists a plethora of tools who aim to help in the process of developing complex domain models, but Domain Driven Design (DDD) is not one if them. \parencite{Bankes, 10.1007/978-3-319-24309-2_33} DDD is more of a thought process and methodology to follow every step of the process. \parencite{EvansEric2004Dd:t} In \emph{Domain-driven design reference: definitions and patterns summaries} do \textcite{evans_2015} describe DDD trough three core ideas:

\begin{itemize}
  \item Focus on the core domain.
  \item Explore models in a creative collaboration of domain practitioners and software practitioners.
  \item Speak a ubiquitous language within an explicitly bounded context.
\end{itemize}

The core domain is the part of your product that is most important and often is your main selling point compared to other similar products. \parencite{millett_2015} A discussion and even possible a documentation describing the core domain is something that will help the development of the product. The idea is to keep everybody on the same track heading in the same direction. \parencite{EvansEric2004Dd:t}

The second idea is to explore and develop every model in collaboration between domain practitioners, who are experts in the given domain, and software developers. This ensures that important knowledge needed to successfully develope the product is communicated back and forth between the two parties. \parencite{millett_2015} The third idea is important to enable and streamline the second. By using a ubiquitous language will miscommunication between domain and software practitioners be minimized and the collaboration between the two parties can instead focus on the important parts which is to develop the product. \parencite{evans_2015}

\textcite{evans_2015} do as well argue about the weight of clearly defining the bounded contexts for each defined model, and this needs to be done in the ubiquitous language created for the specific product. The need of this exists because of the otherwise great risk of misunderstandings and erroneous assumptions in the collaborations between the different models. \parencite{millett_2015}


\subsection{Domain Driven Security}

