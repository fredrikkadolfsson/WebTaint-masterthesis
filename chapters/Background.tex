\chapter{Background}


\section{Web Applications}

\section{Injection}
\subsection{Cross-site Scripting}
\subsection{SQL}

\section{Taint Propagation}
\subsection{Dynamic}
\subsection{Static}

\section{Domain Driven Design}
There exists a plethora of tools who aim to help in the process of developing complex systems, but Domain Driven Design (DDD) is not one if them. DDD is more of a thought process and methodology to follow trough every step of the process. \parencite{EvansEric2004Dd:t} In \emph{Domain-driven design reference: definitions and patterns summaries} do \textcite{evans_2015} describe DDD trough three core ideas:

\begin{itemize}
  \item Focus on the core domain.
  \item Explore models in a creative collaboration of domain practitioners and software practitioners.
  \item Speak a ubiquitous language within an explicitly bounded context.
\end{itemize}

The core domain is the part of your product that is most important and often is your main selling point compared to other similar products. \parencite{millett_2015} A discussion and even possible a documentation describing the core domain is something that will help the development. The idea is to keep everybody on the same track heading in the same direction. \parencite{EvansEric2004Dd:t}

The second idea is to explore and develop every model in collaboration between domain practitioners, who are experts in the given domain, and software developers. This ensures that important knowledge needed for the product is communicated to the developers. \parencite{millett_2015} The third idea is important to enable and streamline the second. By using a ubiquitous language will miscommunication between domain and software practitioners be minimized and the collaboration between them can instead focus on the important parts which is to develop.  \parencite{evans_2015}
 
\subsection{Domain Driven Security}

